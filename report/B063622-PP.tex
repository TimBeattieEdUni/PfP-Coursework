%%%%%%%%%%%%%%%%%%%%%%%%%%%%%%%%%%%%%%%%%%%%%%%%%%%%%%%%%%%%%%%%%%%%%%%%%%%%%%%
% MSc HPC
% Performance Programming
% Coursework
% Exam no. B063622
%
% Report.
%


%%%%%%%%%%%%%%%%%%%%%%%%%%%%%%%%%%%%%%%%%%%%%%%%%%%%%%%%%%%%%%%%%%%%%%%%%%%%%%%
% Packages.
%
\documentclass[11pt, oneside]{article}   % use "amsart" instead of "article" for AMSLaTeX format
\usepackage{geometry}                    % See geometry.pdf to learn the layout options. There are lots.
\geometry{letterpaper}                   % ... or a4paper or a5paper or ... 
\usepackage[parfill]{parskip}            % Activate to begin paragraphs with an empty line rather than an indent
\usepackage{graphicx}                    % Use pdf, png, jpg, or eps§ with pdflatex; use eps in DVI mode
                                         % TeX will automatically convert eps --> pdf in pdflatex		
\usepackage{epstopdf}
\usepackage{amssymb}
\usepackage{listings}


% load .eps files created by GnuPlot for epstopdf to convert to .pdf
\DeclareGraphicsExtensions{.eps}


%%%%%%%%%%%%%%%%%%%%%%%%%%%%%%%%%%%%%%%%%%%%%%%%%%%%%%%%%%%%%%%%%%%%%%%%%%%%%%%
% Components.
\title{Performance Programming \\ Coursework}
\author{B063622}
\date{\today}


%%%%%%%%%%%%%%%%%%%%%%%%%%%%%%%%%%%%%%%%%%%%%%%%%%%%%%%%%%%%%%%%%%%%%%%%%%%%%%%
% The document itself.
%
\begin{document}

\pagenumbering{gobble}   % No page number on title page.
\maketitle

\newpage

\pagenumbering{roman}   % Preamble pages numbered in Roman numerals.
\tableofcontents

\newpage

\pagenumbering{arabic}  % Body of report numbered in Arabic.

\section{Introduction}
The following report describes the process of optimising a program which runs a molecular dynamics calculation.
The system contains a number of particles which move under both an inverse square attraction force and a viscosity in the system.
Several modifications to the code are described, including several which made performance worse and were discarded from the final code.
An analysis of each modification's effects on the code's performance is also presented.

\newpage

\section{The Initial Code}
Debug flag in makefile.
Large arrays, powers of 2, cache lines.
Nested for loops with fastest-moving index outermost.
Function calls for simple calcs.
Combinable loops.

\section{The Given Code}
First step to examine the code as-is.
Brief overview of initial code.
Initial runs unmodified - (193.724736 + 193.554321 + 193.863500 + 193.568124 + 193.937730) / 5- very consistent timings at ~193.7 seconds to within 1/2 %
Data arrays large powers of 2.

\section{Testing for Correctness}
Needed test to ensure correct mods - many mods.
Test program compared output within a given delta.
Didn't detect NaNs - had to add this - list code.
Needed collisions, empirically found they started at 162 (show output), so original output at 200 used as input.
Original program used to produce output 100 steps after 200th in input.

\section{Repeatable work process}
Modify code - small mods to isolate effects.
Run code.
64 cores - isolate.
Verify output
Measure performance - 2 runs, consistent.
Store output data, 2 morar outputs, and output from test for each modification.
Revert change if performance worse.

\section{Initial Steps}
First modification: remove -g flag.
Debug symbols - increase size of code and reduces performance.
107.066320s and 107.309783 
Consistent timing found using 64 cores, so 2 runs enough.
If runs varied, planned to increase number of runs.  Didn't vary.

Profiling run done, but unsurprisinbly all time spent in function evolve() as it contained all the loops.  Could have separated loops into functions, but obvious candidates (3-nested fors) so worked on them.

\section{Compiler optimisation flags}
Easy initial option: try adding compiler optimisation flags  O1-4 tried as well as–fast –Mipa=fast,inline  as recommended by th PGI Compiler User Guide. \cite{ref:PgiCC}  "If you want to get started quickly with optimization, a good set of options to use with any of the PGI compilers is -fast -Mipa=fast,inline."
O1: (only 1 run) 125.075299
O2: 127.319100
O3: 127.375565 127.023777
O4: 125.678674 125.906789
-fast -Mipa=fast,inline: 
Found that all optimisation flags actually increased run time.  Clear that compiler optimisation not a limiting factor in current code.
Left -fast -Mipa=fast,inline in place as expected to solve other problems - compiler optimisation flags generally make an improvement.

\section{Conclusion}
This section is the conclusion.

\begin{thebibliography}{100}

\bibitem{ref:PgiCC} {\em 2015 PGI Compiler User Guide} NVIDIA Corporation. pp 24-26.

\end{thebibliography}

\end{document}